\documentclass{article}
\usepackage{amsmath,amssymb,graphicx}
\title{E8-Fractal-Driven Quantum Gravitational Computing 👾⚛️🔥}
\author{Travis D. Jones}
\date{July 20, 2025}

\begin{document}

\maketitle

\section{Abstract}
E8 lattice projects to 4D quasicrystal via golden ratio ($\phi$) rotations of tetrahedra, yielding icosahedral symmetry and Fibonacci icosagrid for emergent 3D spacetime, unifying forces and resolving singularities.\cite{web0}\cite{web1}\cite{web2}\cite{web3}\cite{web8}

\section{Introduction}
E8, 248D Lie group, unifies symmetries.\cite{web10} Projects to 4D Elser-Sloane quasicrystal: $\Pi = -\frac{1}{\sqrt{5}} \begin{pmatrix} \tau I & H \\ H & \sigma I \end{pmatrix}$.\cite{web10}\cite{web11}

\section{Golden Ratio Rotations}
Tetrahedra rotate by $\arccos((3\phi-1)/4)$, forming 20-clusters with icosahedral (H3) symmetry.\cite{web11}\cite{web12}

\section{Fibonacci Icosagrid}
10-plane multigrid: $x_{n,N} = T(N + \alpha + \mu/\rho^N + \beta)$, $\rho=\mu=\phi$, composing 5 tetragrids.\cite{web10}\cite{web12}\cite{web5}\cite{web8}

\section{Emergent Spacetime}
3D slices embed tetrahedra, discrete quasiperiodic spacetime unifies gravity/gauge, resolves singularities via symmetric extensions.\cite{web11}\cite{web3}\cite{web6}

\section{Simulation}
6D Josephson junction integrates Tetbit with E8 roots.

\bibliographystyle{plain}
\bibliography{refs} % Assume refs.bib with web citations

\end{document}
